\documentclass[a4paper,12pt]{article}
\usepackage[english]{babel}
\usepackage[utf8]{inputenc}

%
% For alternative styles, see the biblatex manual:
% http://mirrors.ctan.org/macros/latex/contrib/biblatex/doc/biblatex.pdf
%
% The 'verbose' family of styles produces full citations in footnotes, 
% with and a variety of options for ibidem abbreviations.
%
\usepackage{csquotes}
\usepackage[style=verbose-ibid,backend=bibtex]{biblatex}
\bibliography{sample}

\usepackage{lipsum} % for dummy text

\title{FOAR705 Learning Journal}

\author{John Hundley - 45072973}

\date{\today}

\begin{document}
\maketitle

\section{WEEK 2 (Data Carpentry):}

(7/08/19)

Intention: Investigate problems with Excel spreadsheets

Action: Downloaded spreadsheets we were asked to. Opened the spreadsheet files. Noticed the data was everywhere. There were two different tabs. There were colour coded cells. Personal and contextual markers present. Zeroes are not filled in. Titles are poor, not created appropriately for Excel. Metadata not properly stored (should be stored in a separate document). The number of livestock and category of livestock are in the one column (should be all separate columns). The information would be hard to understand for people who are foreign to the document. 

Result: Learned what a poor spreadsheet looks like. 

\section{WEEK 2 (Cloudstor/Slack/Github):}

(7/08/19)

Intention: To learn the essentials of the Cloudstor, Slack, and Github programs. 

Action. Started with Cloudstor. Signed up through the university portal. Had a look around for a few minutes. Seems like a standard cloud data repository service. Nothing new. Then, began to play around on Github. Signed up with my university account. It was very confusing to begin. Jeremy showed me some of the basics. Learnt how to ‘pull’ a document, commit changes, and create different branches to a master document. Read the basic instructions. Seemed to make sense. Then was invited onto the Slack page. Seemed pretty intuitive just like a host of other message boards. No issues.

Result: Basic understanding of the three programs. 

\section{WEEK 3 (Overleaf):}

\item(12/08/19)

Intention: To learn the basics of Overleaf.

Action: Registered for an account on Overleaf. Read the 30-minute guide. Started writing out the scoping exercise on Overleaf. Selected a Macquarie University template. Changed bits and pieces to ensure it was the appropriate format. For example, I deleted the ‘abstract’ and ‘authors’ section. I replaced them with relevant headings. Accidently deleted the title and my name. Could not work out a way to undo the error. ‘Fixed’ the problem by inserting the title in a ‘section’ heading. Looked a little dodgy but did the job. Realised I was typing on the ‘source’ section instead of the ‘rich text’ section. Switched over. Finished for the time being. Unsure how to save the document. Downloaded it as a PDF just in case. 

Result: Slightly odd looking document but looks suitable enough for submission. 

\item(13/08/19) 

Intention: To finish scoping exercise and submit it on iLearn, Cloudstor, and Github. 

Action: Wrote the rest of the exercise out in word. Transferred it over into Overleaf template after I had finished. No major issues. Saved as a PDF and a zip file. Submitted the PDF on iLearn. Then I created a new folder on Cloudstor and submitted both files. Next, I went onto Github, created a new branch of the original Hundley-Exercises file. Then saved the document there also. 

Result: Three submissions of one single document. No errors. 

\section{WEEK 3 (Data Carpentry):}

(13/08/19)

Intention: To clean the messy spreadsheet file.

Action: Started creating new columns. Entered in correct titles. Condensed the two tabs into one. Fixed the null values. Got rid of the colour coded cells. Having problems with merging cells. Can’t seem to find an easy way to do it.

Result: Finished it with no major issues. Was later told we didn’t actually need to clean the data...

\section{WEEK 4 (Data Carpentry):}

\begin{itemize}
    \item 

(21/08/19)

\item Intention: Fix the dates file.

\item{Action: Worked with Jeremy, Jack, Lauren, and Ellen. Helped each other figure out how to clean it up. Opened dates file. Created three new tables according to day/month/year. Used the extract function. Completed. No errors. Then moved onto the Quality Assurance lesson. Pocked the years-liv column to insert a numerical range. Thought that 1-99 is reasonable. Then picked the respondent-wall-type column to type in a message through the input message box. Typed ‘this is the type of wall’. Submitted it. No errors. Hovered mouse over the column and saw that the message popped up. Moved onto the Exporting Data lesson. Followed instructions and saved the file in csv format. No errors.}

\item{Result: Finished the data carpentry lessons for the week with no major errors.}

\end{itemize}

\section{WEEK 4 (Overleaf):}

\begin{itemize}
    \item 

(21/08/19)

\item Intention: To write out the scoping exercise. 

\item{Action: Wrote out the scoping exercise with some of the people mentioned above. Wrote it out in Word first and then copied and pasted onto Overleaf. Seems easier this way. Selected a Macquarie University template. Pasted text onto Overleaf. Fixed headings and numbers. Downloaded file as PDF and zip. No errors.} 

\item{Result: Finished scoping exercise with no errors.}

\end{itemize}

\section{WEEK 5 (Overleaf):}

\begin{itemize}
    \item 

(27/08/19)

\item {Intention: To write out the elaboration exercise and understand how Voyant can assist this project.}

\item {Action: Figured out how to use the 'item' and 'itemize' command. Now my structure and paragraphs looks neater. Having problems with the Week 3 Data Carpentry entry. Am doing the exact same thing I have done with the previous sections yet it is not following the commands. Will now commit the changes to show the problem... Tried to fix it and now have made it worse. Not sure what the solution here is. Ended up going back through and fixing it manually.}

\item {Result: Made a major error. Albeit, now the document has been returned to its slightly original form.}

\end{itemize}

\section{WEEK 5 (The Unix Shell):}

\begin{itemize}
    \item 

(28/08/19)

\item Intention: To complete the shell episodes. Firstly, navigating files and directories. 

\item Action: Opened GitBash. Typed in ls. The directory was listed. Typed in ls-l. The directory was listed in a much longer format. Tried entering the -h command. Did not work. Entered 'ls-h'. No result. Fixed it. Entered the command in with spaces instead. Entered 'ls -l -h' and it produced the long list again. 

Next, I entered 'ls -R -t'. Produced a huge list. Cannot read all of it. The solution section on the activity says that the files are sorted according to the last changes made. 

Then, I typed in: 'ls -F Desktop/data-shell'. The result was "ls: cannot access 'Desktop/data-shell': No such file or directory". Don't know what the problem is. 

Entered 'ls -F -a'. Produced a legible list. 

Entered 'cd Desktop/data-shell/data'. Resulted in 'bash: cd: Desktop/data-shell/data: No such file or directory'. Once again, I am not sure what the problem is. The data shell folder is installed on my desktop.

For the Absolute Vs Relative Paths exercise: I guessed No.8 - I was correct.

For the Relative Path Solution exercise: I guessed No.4 - I was also correct. 

For the Reading Comprehension exercise. I picked No.3.

\item Result: Many errors. I am still unsure how to solve the problems I encountered. The 'Desktop-data-shell' command did not cooperate.

\item Intention: To complete the Working With Files episode. 

\item Action: Opened NotePad. Typed in 'Palm Trees'. Saved the file as 'palm-trees'. Closed the NotePad. Entered 'touch palm-trees'. Nothing happened. Entered 'ls -l'. Resulted in the list. Found 'palm-trees' file in the list. Although there is none of the information the lesson says it should have. It is not displaying its size. 

For the 'Renaming Files' exercise: I picked option 2. Then tried it with the 'palm-trees' file. Nothing happened. 

For the 'Moving and Copying' exercise: I had no idea.

Realised the 'palm-trees' file was saved in the My Documents folder. Moved it to the desktop. Now I will see what happens when I enter the 'rm' command. Did so, and still...nothing happened. 

In the 'Copying Multiple Frames' exercise, entered: 'cp desktop data-shell'. The result was 'cp: -r not specified; omitting directory 'desktop''

In the 'List File Names and Acheive a Pattern' exercise: I did not know. 

For the remaining three exercises I also did not know the correct answers. I am going to see if someone can show me where I am going wrong in producing the errors I am getting.

\item Result: Multiple errors again. There must be something major that I am missing. Just not sure what...

\section{WEEK 6 (The Unix Shell):}

\begin{itemize}
\item
(4/09/19)

\item Intention: To figure out why all previous attempts have not resulted in any actionable commands and subsequent results. 

\item Action: Figured out the data-shell file was stored on OneDrive/Desktop. So I entered /OneDrive/Desktop. It then confirmed I was in Desktop. Then I entered cd data-shell. Then it confirmed I was in data-shell. Problem solved. 

\item Errors: All of last week I was trying to access the data-shell file on the desktop without realizing that Windows 10 automatically saves everything (even the desktop!) to OneDrive. Now I can access the data-shell file through the terminal properly. 

\item Result: Now I can continue with the exercises. 


\item \textbf{Intention: Run the command 'wc* .pdb' (9:04AM).}

\item Action: Ran the command. 

\item Result: Listed the details of the files. 

\item Errors: None

\item \textbf{Intention: Run the command 'wc -1' (9:08AM)}. 

\item Action: Entered the command.

\item Result: Nothing happened. Entered ctrl + c to escape. 

\item Errors: The website says that I did not give it a valid input. 

\item \textbf{Intention: To run 'lengths.text' (9:17AM).}

\item Action: Ran it. 

\item Result: 'No file in such directory found.' 

\item Errors: Could not find the file. Not sure what the problem is here. 

\item \textbf{Itention: Sort the files in 'molecules' folder (9:23AM).}

\item Action: Entered 'sort -n lengths.txt'. 

\item Result: No result. Could not locate the file. 

\item Errors: No file or directory found. 

\item \textbf{Intention: To use the '|' function} (9:41AM). 

\item Action: Entered 'wc -l *.pdb | sort -n'. 

\item Result: No such file or directory.

\item Errors: As above. 

\item \textbf{Intention: To figure out why animals.txt or molecules file won't cooperate (2:27PM).}

\item Action: Entered 'cat animals.txt | head -n 5'. It worked. It produced the data in the folder. I think it might be the '|' function. Maybe it connects the commands or something? I entered the specific file names with a specific command and it was successful.

\item Result: Understood why the previous commands were not working. I have to name a specific file or folder (seems so obvious now!). 

\item \textbf{Intention: To sort the animals.txt file with head/tail/sort/ > commands. (2:31PM).}

\item Action: Entered cat animals.txt | head -n 5 | tail -n 3 | sort -r > final.txt. Did not work. Could not locate file. 

\item Errors: As above.

\item Result: No result. 

\item \textbf {Intention: To use the 'cut' function (2:37PM).}

\item Action: Entered 'cut -d , -f 2 animals.txt'

\item Result: Produced the list of 'deer/rabbit/raccoon etc...' No errors. 

\item \textbf{Intention: To use the uniq command (2:40PM).}

\item Action: Entered 'cut -d , -f 2 animals.txt | sort | uniq'

\item Result: Sorted the list of animals alphabetically with the second column gone. No errors. 

\item \textbf{Intention: To test the wildcard expression (2:51PM).} 

\item Action: Entered 'cd north-pacific-gyre/2012-07-03'. Then *[AB].txt. It said the command was not valid. Then I looked at the solution on the exercise and tried the solutions provided. Entered 'ls *A.txt' and 'ls *B.txt'. It gave me the lists of the data according to the chosen letter. 

\item Result: No major errors. Followed the instructions and it was all good. 

\item \textbf {Intention: To show the output of data code loop: 'for datafile in *.pdb
> do
>    ls *.pdb
> done' (3:26PM)}

\item Action: Entered in the code. It produced the list of the outputs but sorted in matching columns of six. 

\item Result: Produced the list. No errors.

\item \textbf {Intention: To limit a set of files (3:31PM).}

\item Action: Entered 'for filename in c*
> do
>    ls filename
> done'

\item Result: Produced the cubane.pdb file. No errors.

\item \textbf{Intention: To learn how to save a file in a loop (3:34PM).}

\item Action: (Note: Dollar signs have been deleted from transcript here to avoid issues in Overleaf.) Firstly, I entered 'for alkanes in *.pdb
do
    echo alkanes
    cat alkanes > alkanes.pdb
done'. Then, secondly, entered 'for datafile in *.pdb
do
    cat datafile >> all.pdb
done'

\item Result: On the first command it produced the list of molecules in a .pdb format? Not entirely sure. I think they are all saved in the alkanes file? With the second command it did not do anything. 

\item \textbf{Intention: Testing commands via 'echo' prompt. (10:26AM)}

\item Action: cd into data file. Then ran 
 "for file in *.pdb
> do
>   echo "analyze file > analyzed-file"
> done". Showed what it would have done if it had analysed the file. 

\item Result: It ran the command correctly. No errors.

\item \textbf{Intention: To create a new file within Git-Bash (10:40AM.}

\item Action: Keyed in 'nano' into the molecules file. It brought me to a new window. Wrote the required text then saved it. Exited out and checked within the GUI if existed and it did.

\item Result: Created a file in Git-Bash for the first time. No errors.

\item \textbf{Intention: Replace 'octane.pdb' with the variable "$1$" (10:50AM):}

\item Action: Entered in the command. It produced a list of atoms. It seems to have worked.

\item Result: Entered a command into the nano text editor and check it within the main git-bash window and it worked. No problems. 

\item \textbf{Intention: To use the 'grep' command (11:03AM).} 

\item Action: Entered grep 'not' haiku.txt. It found the word 'not'. Did the same for the -w, -n, -i, and -v.

\item Result: It found the commands I entered. Learnt how to use grep.

\item \textbf {Intention: to use the 'find' command (11:19AM).}

\item Action: Entered 'find .' in the writing folder file. It produced a list. Then added the command '- type d'. It narrowed the list down. Also tried it with '-type' and '-name'.

\item Results: The find command found the required files according to the command. For example, when I entered in ' find. , name '*.txt'' it  found four files with txt in their name. In addition, finished the novice shell units. 

\end{itemize}

\section{WEEK 6 (Elaboration 2 Testing):}

(4/09/19)

\begin{itemize}

\item \textbf {Intention: To test search function on Voyant. I will search three documents for a single term 'Dionysian'. I want Voyant to provide data on this term within the chosen documents (12:20PM).} 

\item Action: Uploaded the three documents to Voyant's search engine. The engine then produced a new window where the three documents were analysed. I then selected 'Dionysian' from the 'Cirrus' section. It informed me that 'Dionysian' was used 118 times across the three papers. I then went to the 'Terms' window. I selected 'Dionysian'. From there I could see that in the 'Trends' window across the screen, the relationship between that term and the three texts. It gave me information on word count, frequency, and distribution. Then, in the bottom right hand corner, I was able to sort the term by correlative and statistical significance. 

\item Result: Voyant appears to search multiple documents for a single term effectively. However, its layout and user-interface is slightly confusing. Also, it does not offer a textual analysis as hoped for. 

\item \textbf{Intention: Assessing whether Macquarie Universty library system grants access to the first random three files after a certain search term.}

\item Action: Searched 'Kant's beautiful/sublime'. Returned with multiple texts. Chose the first three. All were accessible. The journals were subscribed to by Macquarie University. The first was 'ROAD: Directory of Open Access Scholarly Resources'. The second was 'Taylor and Francis Online'. The third was 'JSTOR Arts and Sciences'. This shows a high probability that Macquarie's system has access to most high quality journals. 

\item Result: Success. There were no issues accessing the journals. Macquarie University library system is clearly an appropriate tool to use when conducting research. 

\item \textbf {Intention: To test whether Zotero can produce properly formatted bibliographic entries.}

\item Action: Dowloaded Zotero and Zotero Connector. Then accessed an article by Vandenabeele (2003) on Macquarie University Library system. Clicked the 'Save to Zotero' option on newly installed toolbar. The document was then saved to my Zotero library. I opened Zotero and accessed the document and it had all the necessary bibliographic information.

\item Result: Success. Zotero is an easy reference generating tool. I was considering using OneNote if it was too difficult to use but it proved simple.  
    
\item \textbf{Intention: To test whether texts can be uploaded to Zotero from offline sources with full bibliographic details.}

\item Action: Tried to import a file from my hard-drive. It said the action was not supported because the file was in the wrong format. The file was in a PDF format. It says it only supports files in RIS, Bib(La)Tex, and MODS that can be imported. I then tried the 'Link to File' on the same file. It appeared in Zotero, however, it was completely lacking any details other than the article name and author. This is a significant problem for any files that I upload whilst offline. Instead, I will have to manually enter in the details. Unfortunately, this is very similar to simply writing out the bibliographic details in Word. 

\item Result: Zotero cannot provide full bibliographic details when a source is uploaded offline. The source has to be added to the Zotero collection when accessed online straight away otherwise it will not work.

\end{itemize}


\section{WEEK 7 (Open Refine):}

(12/09/19)

\begin{itemize} 

\item \textbf{Intention: Trim leading and trailing white space (4:00PM).}

\item Action: Edited the 'burntbricks' and 'mudduab' in the respondent-wall-type section. Deleted the white space in front of the word. 

\item Result: It combined the two different categories into one. ' burntbricks' was edited to be 'burntbricks'. 

\item \textbf{Intention: To filter the table so I can work on a subset of data instead of the whole thing (4:08PM).}

\item Action: Hit 'text filter' on the 'respondent-roof-type' section. Then typed 'mabat' into the side bar. It narrowed the result to the 'mabatisloping' and 'mabatipitched'. Then I typed in 'mabatiptiched' and it only showed that term. 

\item Result: What I typed into the text box showed in the corresponding column. 

\item \textbf {Intention: To include/exclude data (4:20PM)}

\item Action: Hit the 'text facet' on the 'respondent-roof-type' and played around with the include/exclude tools.

\item Result: By selecting include on 'grass' it only included 'grass' on the columns on the right hand side.

\item \textbf{Intention: To sort 'GPS altitude' data (4:25PM).}

\item Action: Clicked the 'sort' option on the 'GPS altitude' column. Then sorted it by numbers.

\item Result: The first ten results contained a zero. The data is probably missing given the rest of the altitude data is around the mid-high 600s. 

\item \textbf{Intention: To sort multiple columns (4:34PM) }

\item Action: Move the 'village' column to the end. Lined up the longitude and latitude columns with it. Then sorted the 'interview date' by date. It looked like 49 was Chirodzo. The GPS coordinates and dates lined up with each other. Then changed the name 49 into Chirodzo. 

\item Result: Now there are only three villages, God, Ruaca, and Chirodzo. 


\end{itemize}


\end{document}