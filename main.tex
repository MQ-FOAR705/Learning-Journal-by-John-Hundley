\documentclass[a4paper,12pt]{article}
\usepackage[english]{babel}
\usepackage[utf8]{inputenc}

%
% For alternative styles, see the biblatex manual:
% http://mirrors.ctan.org/macros/latex/contrib/biblatex/doc/biblatex.pdf
%
% The 'verbose' family of styles produces full citations in footnotes, 
% with and a variety of options for ibidem abbreviations.
%
\usepackage{csquotes}
\usepackage[style=verbose-ibid,backend=bibtex]{biblatex}
\bibliography{sample}

\usepackage{lipsum} % for dummy text

\title{FOAR705 Learning Journal}

\author{John Hundley - 45072973}

\date{\today}

\begin{document}
\maketitle

\section{WEEK 2 (Data Carpentry):}

(7/08/19)

Intention: Investigate problems with Excel spreadsheets

Action: Downloaded spreadsheets we were asked to. Opened the spreadsheet files. Noticed the data was everywhere. There were two different tabs. There were colour coded cells. Personal and contextual markers present. Zeroes are not filled in. Titles are poor, not created appropriately for Excel. Metadata not properly stored (should be stored in a separate document). The number of livestock and category of livestock are in the one column (should be all separate columns). The information would be hard to understand for people who are foreign to the document. 

Result: Learned what a poor spreadsheet looks like. 

\section{WEEK 2 (Cloudstor/Slack/Github):}

(7/08/19)

Intention: To learn the essentials of the Cloudstor, Slack, and Github programs. 

Action. Started with Cloudstor. Signed up through the university portal. Had a look around for a few minutes. Seems like a standard cloud data repository service. Nothing new. Then, began to play around on Github. Signed up with my university account. It was very confusing to begin. Jeremy showed me some of the basics. Learnt how to ‘pull’ a document, commit changes, and create different branches to a master document. Read the basic instructions. Seemed to make sense. Then was invited onto the Slack page. Seemed pretty intuitive just like a host of other message boards. No issues.

Result: Basic understanding of the three programs. 

\section{WEEK 3 (Overleaf):}

(12/08/19)

Intention: To learn the basics of Overleaf.

Action: Registered for an account on Overleaf. Read the 30-minute guide. Started writing out the scoping exercise on Overleaf. Selected a Macquarie University template. Changed bits and pieces to ensure it was the appropriate format. For example, I deleted the ‘abstract’ and ‘authors’ section. I replaced them with relevant headings. Accidently deleted the title and my name. Could not work out a way to undo the error. ‘Fixed’ the problem by inserting the title in a ‘section’ heading. Looked a little dodgy but did the job. Realised I was typing on the ‘source’ section instead of the ‘rich text’ section. Switched over. Finished for the time being. Unsure how to save the document. Downloaded it as a PDF just in case. 

Result: Slightly odd looking document but looks suitable enough for submission. 


(13/08/19) 

Intention: To finish scoping exercise and submit it on iLearn, Cloudstor, and Github. 

Action: Wrote the rest of the exercise out in word. Transferred it over into Overleaf template after I had finished. No major issues. Saved as a PDF and a zip file. Submitted the PDF on iLearn. Then I created a new folder on Cloudstor and submitted both files. Next, I went onto Github, created a new branch of the original Hundley-Exercises file. Then saved the document there also. 

Result: Three submissions of one single document. No errors. 

\section{WEEK 3 (Data Carpentry):}

\begin{itemize}

(13/08/19)

\item{Intention: To clean the messy spreadsheet file.}

\item{Action: Started creating new columns. Entered in correct titles. Condensed the two tabs into one. Fixed the null values. Got rid of the colour coded cells. Having problems with merging cells. Can’t seem to find an easy way to do it.}

\item{Result: Finished it with no major issues. Was later told we didn’t actually need to clean the data…}

\end{itemize}

\section{WEEK 4 (Data Carpentry):}

\begin{itemize}
    \item 

(21/08/19)

\item Intention: Fix the dates file.

\item{Action: Worked with Jeremy, Jack, Lauren, and Ellen. Helped each other figure out how to clean it up. Opened dates file. Created three new tables according to day/month/year. Used the extract function. Completed. No errors. Then moved onto the Quality Assurance lesson. Pocked the years-liv column to insert a numerical range. Thought that 1-99 is reasonable. Then picked the respondent-wall-type column to type in a message through the input message box. Typed ‘this is the type of wall’. Submitted it. No errors. Hovered mouse over the column and saw that the message popped up. Moved onto the Exporting Data lesson. Followed instructions and saved the file in csv format. No errors.}

\item{Result: Finished the data carpentry lessons for the week with no major errors.}

\end{itemize}

\section{WEEK 4 (Overleaf):}

\begin{itemize}
    \item 

(21/08/19)

\item Intention: To write out the scoping exercise. 

\item{Action: Wrote out the scoping exercise with some of the people mentioned above. Wrote it out in Word first and then copied and pasted onto Overleaf. Seems easier this way. Selected a Macquarie University template. Pasted text onto Overleaf. Fixed headings and numbers. Downloaded file as PDF and zip. No errors.} 

\item{Result: Finished scoping exercise with no errors.}

\end{itemize}

\section{WEEK 5 (Overleaf):}

\begin{itemize}
    \item 

(27/08/19)

\item {Intention: To write out the elaboration exercise and understand how Voyant can assist this project.}

\item {Action: Figured out how to use the 'item' and 'itemize' command. Now my structure and paragraphs looks neater. Having problems with the Week 3 Data Carpentry entry. Am doing the exact same thing I have done with the previous sections yet it is not following the commands. Will now commit the changes to show the problem...}

\end{itemize}


\end{document}